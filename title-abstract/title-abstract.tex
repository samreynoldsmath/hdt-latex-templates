%%%%%%%%%%%%%%%%%%%%%%%%%%%%%%%%%%%%%%%%%%%%%%%%%%%%%%%%%%%%%%%%%%%%%%%%%%%%%%%%
\documentclass[12pt,notitlepage]{report}

%%%%%%%%%%%%%%%%%%%%%%%%%%%%%%%%%%%%%%%%%%%%%%%%%%%%%%%%%%%%%%%%%%%%%%%%%%%%%%%%
% PACKAGES
%%%%%%%%%%%%%%%%%%%%%%%%%%%%%%%%%%%%%%%%%%%%%%%%%%%%%%%%%%%%%%%%%%%%%%%%%%%%%%%%

% package for projects with multiple files
\usepackage{subfiles}

% formatting packages
\usepackage[utf8]{inputenc}
\usepackage[margin=3cm]{geometry}
\usepackage{titlesec}	% titles
\usepackage{fancyhdr}	% headers and footers
\usepackage{authblk}	% author and affiliation

% symbol packages
\usepackage{amsmath}
\usepackage{amsthm}
\usepackage{amssymb}
\usepackage{mathtools}

%%%%%%%%%%%%%%%%%%%%%%%%%%%%%%%%%%%%%%%%%%%%%%%%%%%%%%%%%%%%%%%%%%%%%%%%%%%%%%%%
% FORMATTING
%%%%%%%%%%%%%%%%%%%%%%%%%%%%%%%%%%%%%%%%%%%%%%%%%%%%%%%%%%%%%%%%%%%%%%%%%%%%%%%%

\setcounter{Maxaffil}{0}
\renewcommand\Affilfont{\itshape\small}

%%%%%%%%%%%%%%%%%%%%%%%%%%%%%%%%%%%%%%%%%%%%%%%%%%%%%%%%%%%%%%%%%%%%%%%%%%%%%%%%
% MACROS
%%%%%%%%%%%%%%%%%%%%%%%%%%%%%%%%%%%%%%%%%%%%%%%%%%%%%%%%%%%%%%%%%%%%%%%%%%%%%%%%

\input{sams-macros.tex}

%%%%%%%%%%%%%%%%%%%%%%%%%%%%%%%%%%%%%%%%%%%%%%%%%%%%%%%%%%%%%%%%%%%%%%%%%%%%%%%
% FRONT MATTER
%%%%%%%%%%%%%%%%%%%%%%%%%%%%%%%%%%%%%%%%%%%%%%%%%%%%%%%%%%%%%%%%%%%%%%%%%%%%%%%

\begin{document}

\title{
	Noodlemorphisms, Invariant Spaghetti Cohomologies,
	and the Bourdain-Puck Conjecture
}

\date{
	2023 AICM Hot Dog Theory Conference
	\\[12pt]
	\today
}

\author[1]{Rachael Ray}

\affil[1]{American Institute for Culinary Mathematics}

\maketitle

\begin{abstract}
	Noodlemorphisms were introduced by Boyardee and Crunch 1973 as a method of
	constructing pasta chain complexes which preserve certain sauce properties.
	In this talk, we present a new method of constructing noodlemorphisms using
	the theory of invariant spaghetti cohomologies.
	We then use a lemma of McDonald and Sanders to show that the Bourdain-Puck
	conjecture holds whenever the spaghetti cohomology of the pasta chain
	complex is pesto-invariant or is a subcomplex of the alfredo cohomology.
	This is joint work with Guy Fieri.
\end{abstract}

%%%%%%%%%%%%%%%%%%%%%%%%%%%%%%%%%%%%%%%%%%%%%%%%%%%%%%%%%%%%%%%%%%%%%%%%%%%%%%%
\end{document}